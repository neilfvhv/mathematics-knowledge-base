\documentclass[a4paper, 11pt]{article}

\title{Mathematics Knowledge Base}
\author{neilfvhv}

\begin{document}

\maketitle

\section{Number Theory}

    \subsection{Fundamental theorem of arithmetic}
        Any integer $a > 1$ can be factored in a unique way as
        $$
            a = p_1^{a_1} \times p_2^{a_2} \times \cdots \times p_t^{a_t}
        $$
        where $p_1 < p_2 < \ldots < p_t$ are prime numbers and where each $a_i$ is a positive integer.

	\subsection{Euler's function}
		For any integer $n > 1$:
		$$
			n = p_1^{a_1} \times p_2^{a_2} \times \cdots \times p_t^{a_t}
		$$
        $$
            \phi{(n)} =  ((p_1 - 1) \times p_1^{a_1 - 1}) \times (p_1 - 1) \times ((p_2 - 1) \times p_2^{a_2 - 1}) \times \cdots \times ((p_t - 1) \times p_t^{a_t - 1})
        $$
        

\section{Complex Analysis}

    \subsection{Euler's formula}
        For any real number $x$:
        $$
            e^{i x} = \cos{x} + i \sin{x}
        $$
        where $e$ is the base of the natural logarithm, $i$ is the imaginary unit.

\end{document}
